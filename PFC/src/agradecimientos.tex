\chapter{Agradecimientos}

\begin{quotation}
\textit{Porque la felicidad no es la suma de las experiencias individuales, va más allá de éstas y tiene mucho que ver con la percepción y memoria interna de nuestra vida en su conjunto. Por esta razón la gratitud es la clave. Gratitud por lo bueno y lo menos bueno de la vida. El ver la vida como una aventura en la que, por supuesto, no todo es fácil, pero en la que el centro se encuentra en el proceso mismo, no en el objetivo final. Vivir con el sentimiento de que cada día es nuevo. Disfrutar del camino y sacurdirse el polvo después de cada caída, porque, como Edward Diener demostró, pasado cierto tiempo de cualquier tragedia, se suelen recuperar los niveles normales de felicidad de cada persona, siendo lo que más nos cuesta superar la pérdida de un ser querido y la del puesto de trabajo. Experimentar dolor en la vida da hondura al ser. Quedarse apegado al dolor es un sinsentido. En definitiva, cada uno es responsable de ver la botella medio vacía o medio llena de la belleza de la vida.}
\end{quotation}

\begin{flushright}
\work{Excusas para no pensar}, \\ \textit{Eduardo Punset}
\end{flushright}

