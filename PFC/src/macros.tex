% Macros que facilitan las tareas de escritura.
% Basadas en el trabajo de Eivind Uggedal.

% Listados de código fuente
\newcommand{\code}[1]{\texttt{{#1}}}

% Título de los libros
\newcommand{\work}[1]{{\textit{\workfont #1}}}

% Expresiones latinas
\let\latin\term

% Entrecomillado
\newcommand{\q}[1]{``#1''}
	
% Referencia a una figura. Formato: Figura 1.2
\newcommand\figureref[1]{%
Figura~\ref{fig:#1}%
}

% Referencia a una seccion sin numeracion de página. Formato: S. A.1
\newcommand\sectionnopageref[1]{%
\textsection~\ref{section:#1}%
}

%\newcommand\figurepageref[1]{%
%\figureref{#1}
%(p.~\pageref{figure:#1})%
%}

%\newcommand\tableref[1]{%
%Table~\ref{table:#1}%
%}

%\newcommand\tablepageref[1]{%
%\tableref{#1}
%(p.~\pageref{table:#1})%
%}

%\newcommand\sourcecoderef[1]{%
%Source Code Listing~\ref{sourcecode:#1}%
%}

%\newcommand\sourcecodepageref[1]{%
%\sourcecoderef{#1}
%(p.~\pageref{sourcecode:#1})%
%}

% Referencia a un capítulo. Formato: Capítulo 3 (pág. 21)
\newcommand\chapterref[1]{%
Capítulo~\ref{chapter:#1}
(pág.~\pageref{chapter:#1})%
}

% Referencia a una sección del libro. Formato: S 1.2 (pág. 13)
\newcommand\sectionref[1]{%
\textsection~\ref{section:#1}
(pág.~\pageref{section:#1})%
}

% Referencia a un apéndice del libro. Formato: Apéndice A (pág. 14)
\newcommand\appendixref[1]{%
Apéndice~\ref{appendix:#1}
(pág.~\pageref{appendix:#1})%
}