\chapter{Código Fuente} 
\label{appendix:apendice-b}

\section{Organización del código de \tiki{}}
\label{appendix:organizacion-codigo}

\tiki{} es un proyecto de código abierto que ha sido desarrollado completamente en \php{}. La organización del código sigue una distribución precisa y lógica. Aún así, para una persona neófita que se interna por primera vez en el desarrollo de alguna librería o modificación, puede pensar que la estructura del propio proyecto es algo caótica. A continuación, se muestra un listado de los directorios y archivos más importantes de \tiki{} así como una breve descripción de ellos:
 
\begin{itemize}
\item \textit{lib/}: Este directorio guarda todas las librerías principales de \tiki{}, si queremos realizar una modificación en el núcleo del programa este es nuestro sitio.

\item \textit{db/}: En este directorio se guardan las definiciones del esquema \textit{relacional} de \tiki{}. Aquí es donde se aloja el archivo \texttt{local.php}.

\item \textit{db/local.php}: Este archivo guarda la configuración a la base de datos que hayamos instalado en \tiki{}. Si queremos reinstalar todo desde cero, simplemente hay que limpiar la base de datos y borrar este archivo.

\item \textit{templates/}: Este directorio guarda las plantillas programadas en Smarty \cite{web:smarty}.

\item \textit{templates\_c/}: Este directorio sirve para guardar la \textit{caché} de \tiki{}. Normalmente no es útil cambiar nada de aquí (aunque se puede borrar el directorio para obligar a que \tiki{} regenere los datos de nuevo).

\item \textit{img}: Este directorio guarda la identidad de \tiki{}. Es útil por si se quiere cambiar el logotipo que muestra por defecto por uno propio.
\end{itemize}

\section{Listado de Profiles}

Se listan todos los \profiles{} que han sido diseñados para llevar a cabo este proyecto:

\subsection{Profile: Configuración de ALMA}
\label{section:configuracion_alma}

\begin{pyglist}[language=text]
Este profile configura Tikiwiki para trabajar en ALMA.

Crea también los \profiles{} necesarios para crear asignaturas de manera automática.

Se puede ejecutar directamente en Admin > Profiles > Advanced > Profile Tester

Esta sección se encarga de configurar las opciones generales de Tiki. ¡No olvides 
cambiar la contraseña del correo por la contraseña original!

Tampoco olvides copiar el contenido de Crear_Comunidad_de_Practica_Template 
a la página wiki con el mismo nombre!

{CODE(caption="YAML" wrap="1")}
preferences:
  language: es
  feature_wysiwyg:  y
  wysiwyg_default: y
  wysiwyg_optional: y
  wikiplugin_equation: y
  wikiplugin_perspective: n
  wikiplugininline_perspective: n
  wikiplugininline_equation: y
  feature_blogs:  n
  feature_categories:  y
  feature_file_galleries:  n
  feature_forums:  n
  feature_freetags:  n
  feature_help:  n
  feature_trackers:  n
  feature_mytiki:  y
  feature_userPreferences:  y
  feature_profiles: y
  feature_perspective: y
  allowRegister:  y
  min_username_length:  1
  max_username_length:  50
  min_pass_length:  4  
  forgotPass:  y
  cookie_name:  ALMA
  wikiHomePage:  HomePage
  sender_email:  alma@uah.es
  siteTitle:  ALMA
  browsertitle: ALMA
  site_title_location: before
  server_timezone: Madrid/Europe
  display_timezone: Madrid/Europe
  zend_mail_handler: smtp
  default_mail_charset: utf-8
  zend_mail_smtp_user: alma.disciplinar@gmail.com
  zend_mail_smtp_pass: password
  zend_mail_smtp_port: 465
  zend_mail_smtp_security: ssl
  zend_mail_smtp_auth: plain
  zend_mail_smtp_server: smtp.gmail.com
  log_mail: n
  wikiplugin_datachannel: y
  profile_channels: crear_comunidad, tiki://local, Crear_Comunidad_de_Practica_Template, Admins
  sitelogo_title: ALMA (Aula Libre Multidisciplinar Abierta)
  sitelogo_src: img/tiki/tikisitelogo.png
  sitelogo_alt: Alma
{CODE}

Esta sección se encarga de crear las páginas Wiki con el Profile de Crear 
Comunidad de Paractica y su Data Channel correspondiente. Los asigna al grupo 
de Profiles y sólo los admins son capaces de ejecutar y ver esas páginas.

{CODE(caption="YAML" wrap="1")}
objects:
  -
    type: wiki_page
    ref: wiki1
    data:
      name: Crear_Comunidad_de_Practica_GUI
      content: "{DATACHANNEL(channel=crear_comunidad)}nombre_asignatura, 
      Nombre de la comunidad de practica{DATACHANNEL}"
  -
    type: wiki_page
    ref: wiki2
    data:
      name: Crear_Comunidad_de_Practica_Template
      content: Rellenar el contenido de esta pagina con el archivo del mismo nombre
  -
    type: category
    ref: profilecateg
    data:
      name: Profiles
      description: Categoria donde se almacenan los profiles de las comunidades de practica
      items:
       - [ wiki_page, $wiki1 ]
       - [ wiki_page, $wiki2 ]
permissions:
 Admins:
  objects:
   -
    type: category
    id: $profilecateg
    allow:
      - admin
 Registered:
  objects:
   -
    type: category
    id: $profilecateg
    deny: view, edit, view_category, edit_category
 Anonymous:
  objects:
   -
    type: category
    id: $profilecateg
    deny: view, edit, view_category, edit_category
{CODE}

Creamos el menú que permite cambiar de asignatura a los usuarios que estén
registrados en la plataforma.

{CODE(caption="YAML", wrap="1")}
objects:
 -
  type: module
  ref: switcher
  data:
   name: perspective
   position: left
   order: 10
   groups: [ Registered ]
   params:
     title: Asignaturas
{CODE}
\end{pyglist}

\subsection{Profile: Creación Comunidad de Práctica}
\label{section:creacion-comunidad}

\begin{pyglist}[language=text]
!!!!!Creación de una asignatura usando Profiles

{CODE(caption=>YAML,wrap=1)}
mappings:
 Member: $profilerequest:nombre_asignatura$Asignatura sin nombre$ Alumnos
 Lead: $profilerequest:nombre_asignatura$Asignatura sin nombre$ Profesores
objects:
 -
  type: category
  ref: project_root
  data:
   name: $profilerequest:nombre_asignatura$Asignatura sin nombre$
   parent: 0
   items:
    - [ wiki page, $dashboard ]
 -
  type: perspective
  ref: perspective
  data:
   name: $profilerequest:nombre_asignatura$sin nombre$
   preferences:
    category_jail: $project_root
    wikiHomePage: $dashboard
    browsertitle: $profilerequest:nombre_asignatura$Asignatura sin nombre$
    style: fivealive.css
    style_option: kiwi.css
    feature_wysiwyg: y
    wysiwyg_default: y
    wysiwyg_optional: y
 -
  type: wiki_page
  ref: dashboard
  data:
   name: $profilerequest:nombre_asignatura$Asignatura sin nombre$ Homepage
   content: "Bienvenido a tu nueva asignatura"
permissions:
 Member:
  autojoin: y
  allow:
      - view 
      - edit
      - wiki_view_history
      - wiki_view_source
      - minor
      - upload_picture
      - rollback
      - view_category
      - search
      - delete_account
      - group_view
      - group_view_members
      - view_category
      - add_object
      - modify_object_categories
  objects:
   -
    type: category
    id: $project_root
    allow:
      - view 
      - edit
      - wiki_view_history
      - wiki_view_source
      - minor
      - upload_picture
      - rollback
      - view_category
      - search
      - delete_account
      - group_view
      - group_view_members
      - add_object
      - modify_object_categories
   -
    type: perspective
    id: $perspective
    allow: [ perspective_view ]
 Lead:
  autojoin: y
  allow:
      - view 
      - edit
      - wiki_view_history
      - wiki_view_source
      - minor
      - upload_picture
      - rollback
      - view_category
      - search
      - delete_account
      - group_view
      - group_view_members
      - view_category
      - add_object
      - modify_object_categories
      - remove
      - rollback
      - wiki_attach_files
      - admin_attachments
      - view_attachments
      - upload_picture
      - minor
      - rename
      - lock
      - edit_structures
      - edit_copyrights
      - wiki_view_comments
      - wiki_view_ratings
      - wiki_vote_ratings
      - wiki_admin_ratings
      - wiki_view_history
      - use_HTML
  objects:
   -
    type: category
    id: $project_root
    allow:
      - view 
      - edit
      - wiki_view_history
      - wiki_view_source
      - minor
      - upload_picture
      - rollback
      - view_category
      - search
      - delete_account
      - group_view
      - group_view_members
      - view_category
      - add_object
      - modify_object_categories
      - remove
      - rollback
      - wiki_attach_files
      - admin_attachments
      - view_attachments
      - upload_picture
      - minor
      - rename
      - lock
      - edit_structures
      - edit_copyrights
      - wiki_view_comments
      - wiki_view_ratings
      - wiki_vote_ratings
      - wiki_admin_ratings
      - wiki_view_history
      - use_HTML
   -
    type: perspective
    id: $perspective
    allow: [ perspective_view ]
{CODE}
\end{pyglist}

\subsection{Profile: Interfaz para Crear Comunidad de Practica}
\label{section:creacion-comunidad-gui}

\begin{pyglist}[language=text]
Este Profile genera una comunidad de práctica con el nombre que se le especifique:
{DATACHANNEL(channel=crear_comunidad)}
nombre_asignatura, Nombre asignatura
{DATACHANNEL}
\end{pyglist}