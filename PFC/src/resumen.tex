\chapter{Resumen }

El presente trabajo consiste en la aceleración de la gestión que tiene que realizar un administrador o docente en una plataforma de aprendizaje, en nuestro caso \tiki{} \cite{web:tikiwiki}.
Con motivo de esta aceleración se han desarrollado unos \profiles{} \cite{web:explicacion-profiles} (una herramienta de automatización que existe dentro de la plataforma y en la que nos basamos para mejorar la gestión), que permiten generar grupos de comunidades de práctica con tan solo pulsar un botón en lugar de tener que configurar cada elemento uno a uno con el consiguiente ahorro de tiempo que ello conlleva. Los resultados son claros, el administrador o docente de la plataforma puede desarrollar mejor su tarea y administrar de manera más eficiente sus comunidades de práctica.

\textbf{Palabras clave}: \textit{ALMA, TikiWiki, Comunidades de Práctica, Gestión, Profiles}

\chapter{Abstract }

This work involves the acceleration of the management that an administrator or teacher has to perform in a learning platform, in our case \tiki{} \cite{web:tikiwiki}.
As a result of this acceleration, we have developed some \profiles{} \cite{web:explicacion-profiles} (an automation tool that exists within the platform on which we rely to improve the management), which can generate groups of communities of practice with few interactions instead of having to configure each element one by one. The facts are clear, the administrator or teacher of the platform can manage more efficiently their communities of practice.

\textbf{Keywords}: \textit{ALMA, TikiWiki, Communities of Practice, Management, Profiles}

\chapter{Resumen extendido}

\lettrine{A}{ctualmente el mundo} de las nuevas tecnologías ha supuesto una revolución en la manera que tenemos las personas de consumir la información. Podemos afirmar que múltiples sectores de la sociedad ven cada día como muchos de los principios sobre los que han estado asentados de manera tradicional dejan de tener sentido y acaban adaptándose, de una manera u otra, a las nuevas formas de expresión: el sector educativo es uno de ellos. Durante los últimos veinte años, y de manera más acentuada con la aparición y posterior expansión de Internet, dicho sector está buscando maneras de adaptarse al nuevo mundo tecnológico (que es el que impera ahora). Si echamos un vistazo a las posibilidades que nos ofrece Internet como herramienta, nos daremos cuenta de que hay muchas propuestas que tratan, con un mayor o menor grado de éxito, de intentar sustituir por completo el concepto que tenemos de enseñanza \textit{in situ}, donde ésta se produce en un único lugar y momento determinado, por algo más genérico y flexible como es la educación a distancia donde el tiempo y el espacio ya no juegan un papel fundamental.

Otras propuestas, como es el proyecto \alma{}, tratan de no ser tan radicales en su visión y buscan combinar las teorías del aprendizaje social (que basa sus principios en las interacciones que se producen entre las personas) con otras más modernas que implican el uso de plataformas \textit{wikis}. De la unión de estos dos conceptos nace lo que \citeauthor{libro:comunidades-de-practica-wenger} denominaron en su libro \work{Communities of Practice: Learning, Meaning, and Identity} \cite{libro:comunidades-de-practica-wenger-primera-edicion} como \textit{comunidades de práctica}. 

El objetivo principal de las comunidades de práctica es pretender que distintas personas, de diferente índole y origen, trabajen de manera aunada y consensuada en un misma temática. Las plataformas \textit{wikis} se prestan como las herramientas que hacen posible que la información fluya de manera equilibrada entre los diferentes participantes. 

Por lo tanto, y en este caso tan concreto, es muy importante averiguar cuales son las necesidades específicas de \alma{} y las implicaciones que tienen el uso de las comunidades de práctica. De esta manera obtenemos una lista de requisitos que se puede analizar para encontrar, de las diversas herramientas que existen en el mercado, una plataforma que se ajuste a las características particulares de nuestro proyecto. En nuestro caso, y tras examinar las ventajas y desventajas de cada una, el resultado de dicha elección fue \tiki{}.

Pero, utilizando dicha herramienta, el principal problema que nos encontramos a la hora de administrar grandes comunidades de práctica es la propia gestión de éstas. Por ejemplo, cuando un administrador necesita crear una comunidad de práctica dentro de \tiki{}, conlleva utilizar diferentes partes de la herramienta que, a simple vista, no son triviales de manejar. Además, podemos añadir, que es un proceso que puede llegar a ser tedioso y con grandes probabilidades de que el administrador cometa errores. Este hecho, en concreto, impide que la plataforma sea accesible a un docente (que es una persona que usualmente no está versada en conocimientos técnicos pero que, dada la naturaleza de \alma{}, puede poseer un rol de administrador en una comunidad de práctica) y eso es algo que debemos solventar.

¿Qué soluciones existen para paliar este problema en concreto? La respuesta son los \profiles{}\footnote{Aunque la \textsc{rae} sugiere que las palabras de origen extranjero deben de ser traducidas a sus homónimas en castellano si no son extranjerismos crudos, se ha decidido mantener la palabra original del inglés por poseer una mayor connotación tecnológica en el contexto en el que se mueve esta obra. La traducción correcta sería \textit{perfiles}.}. Éstos forman parte del repertorio de herramientas de \tiki{} y se sitúan como los útiles que automatizan y solucionan el problema enunciado arriba así como otros totalmente distintos. Son múltiples las ventajas que aportan los \profiles{} a \tiki{} en general y a \alma{} en particular. Listamos, a continuación, algunas de ellas:

\begin{itemize}
\item Permiten personalizar cualquier aspecto de \tiki{} (de esta manera adaptamos la plataforma a nuestras necesidades).
\item Simplifican la vida del administrador ya que le evita tener que manejar la plataforma para tareas repetitivas. La filosofía de un \profile{} es: \q{Configúralo una vez, ¡ejecútalo tantas veces como quieras!}.
\item Facilitan que \tiki{} sea una plataforma más accesible para ese grupo de personas que no tienen tantos conocimientos avanzados. (Las tecnologías deben de ser integradoras y no excluyentes.)
\item Favorecen la reutilización de los propios \profiles{} ya que éstos no son más que un conjunto de instrucciones que se encuentran escritas en un fichero de \textit{texto plano}.
\end{itemize}

Por lo tanto, saber manejar esta herramienta para implementar las comunidades de práctica de una forma cómoda, sencilla y fiable, se convierte en una necesidad más que en una afición; de la misma manera que cuando queremos viajar hasta París en coche hemos tenido que aprender el funcionamiento del acelerador y del freno (así como otra serie de nociones) para poder llegar al destino de manera segura.

A lo largo de este \pfc{} se describen las claves fundamentales para que un lector interesado en aprender las dos cuestiones principales que tratamos aquí (comunidades de práctica y la automatización de éstas) pueda manejarse con soltura en la plataforma de \tiki{} y más concretamente con los \profiles{}.
La organización de este libro sigue una estructura lógica que permite entender los contenidos de manera sencilla y, sobre todo, didáctica (en la \sectionref{organizacion-documento} se puede leer como se ha estructurado la obra).

El resultado de todo este esfuerzo: tanto del análisis de las necesidades del proyecto, como de la implementación de las comunidades de práctica con \tiki{}, como de la automatización en las tareas de gestión utilizando para ello los \profiles{} y la posterior documentación de todo lo que se ha realizado, ha dado como resultado una implementación real de \alma{} en la que los profesores y los alumnos pueden utilizar la herramienta para crear comunidades de práctica paralelas a las enseñanzas impartidas en un aula.
